% !TEX TS-program = pdflatex
% !TEX encoding = UTF-8 Unicode

% This is a simple template for a LaTeX document using the "article" class.
% See "book", "report", "letter" for other types of document.

\documentclass[11pt]{article} % use larger type; default would be 10pt

\usepackage[utf8]{inputenc} % set input encoding (not needed with XeLaTeX)


%%% PAGE DIMENSIONS
\usepackage{geometry} % to change the page dimensions
\geometry{a4paper} % or letterpaper (US) or a5paper or....


\usepackage{graphicx} % support the \includegraphics command and options

% \usepackage[parfill]{parskip} % Activate to begin paragraphs with an empty line rather than an indent

%%% PACKAGES
\usepackage{booktabs} % for much better looking tables
\usepackage{array} % for better arrays (eg matrices) in maths
\usepackage{paralist} % very flexible & customisable lists (eg. enumerate/itemize, etc.)
\usepackage{verbatim} % adds environment for commenting out blocks of text & for better verbatim
\usepackage{subfig} % make it possible to include more than one captioned figure/table in a single float
\usepackage{cite}
% These packages are all incorporated in the memoir class to one degree or another...

%%% HEADERS & FOOTERS
\usepackage{fancyhdr} % This should be set AFTER setting up the page geometry
\pagestyle{fancy} % options: empty , plain , fancy
\renewcommand{\headrulewidth}{0pt} % customise the layout...
\lhead{}\chead{}\rhead{}
\lfoot{}\cfoot{\thepage}\rfoot{}

%%% SECTION TITLE APPEARANCE
\usepackage{sectsty}
\allsectionsfont{\sffamily\mdseries\upshape} % (See the fntguide.pdf for font help)
% (This matches ConTeXt defaults)

%%% ToC (table of contents) APPEARANCE
\usepackage[nottoc,notlof,notlot]{tocbibind} % Put the bibliography in the ToC
\usepackage[titles,subfigure]{tocloft} % Alter the style of the Table of Contents
\renewcommand{\cftsecfont}{\rmfamily\mdseries\upshape}
\renewcommand{\cftsecpagefont}{\rmfamily\mdseries\upshape} % No bold!



\title{Mapping the Brain: An Introduction to Connectomics\\Progress Report: An Analysis of the Performance of Vesicle Detection Systems}
\author{Gregory Levine, Matt Saltzman, Alex Sharata}
%\date{} % Activate to display a given date or no date (if empty),
         % otherwise the current date is printed 

\begin{document}
\maketitle

\section{Summary}
Now that we are well underway with the task of exploring existing methods of vesicle detection in brain EM data and working to improve them, we have decided to make several modifications to our original goal. Before, our intention was to work almost entirely in Matlab to both analyze the existing algorithm used for vesicle detection that is provided at neurodata.io, and to work to improve it. Now, after working with the existing algorithm, it is clear that it performs well enough to facilitate the end goal of synapse detection, and improvement in vesicle detection would not improve synapse detection greatly.  We will still analyze the effectiveness of the existing method. Instead of improving it, however, we will try to implement a similar function in Python, where one does not yet exist. The success (measured using Precision/Recall) of our new algorithm will be compared to the Matlab method. As of now, analysis of the Matlab algorithm has been completed, and the Python version is underway.

\section{Updated Goals}
New Goals:
\begin{enumerate}
\item Perform Precision/Recall Analysis of the Matlab function for vesicle identification from neurodata.io.
\item Implement a similar function, though perhaps using a different identification technique, in Python.
\item Perform Precision/Recall Analysis of the Python function for vesicle identification that we write
\item Compare the performance of the 2 functions.
\end{enumerate}
\section{Updated Timeline}
\begin{enumerate}
\item The analysis of the Matlab function is already completed. 
\item Matt and Alex are working on establishing a new framework in python with a minimum of one feature (either vesicle-blob or rough circles), and making sure it works in the pre-classifier stage before progressing. (1/19)
\item All members will work to perform precision/recall analysis on the new framework. (1/20)
\item Poster and Paper will be finished. (1/22)
\end{enumerate}
\end{document}

